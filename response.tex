\documentclass[12pt,a4paper,preprint]{article}


\usepackage{setspace}
\usepackage{xcolor}
\usepackage{amssymb}
\usepackage{softdev}
\usepackage{quoting}
\usepackage{xspace}
\usepackage[many]{tcolorbox}
\newtcolorbox{blockquote}{%
  sharp corners,
  colback=gray!5,
  colframe=gray!60,
  leftrule=2pt,
  enhanced,
  before skip=6pt, after skip=18pt,
  boxrule=0pt,
  left=6pt, right=4pt,
  borderline west={2pt}{0pt}{gray!60},
  fontupper=\scriptsize\sffamily\selectfont
}


\setlength{\parindent}{0pt}
\setlength{\parskip}{2.0ex plus0.5ex minus0.2ex}

%\newcommand\laurie[1]{\mynote{Laurie}{#1}}
%\newcommand\jake[1]{\mynote{Jake}{#1}}

\newcommand\Egcrc{E$_\textrm{gcvs}$\xspace}
\newcommand\Eelision{E$_\textrm{elision}$\xspace}
\newcommand\Epremopt{E$_\textrm{premopt}$\xspace}

\begin{document}

\date{}  % This removes the date
\title{Response to Reviewer Feedback}
\maketitle

We thank the reviewers for their thoughtful reviews! While our
\texttt{revisions R2} submission contains a PDF diff that shows prose
modifications, we recognize that many review comments centered on our
evaluation approach. Since \texttt{latexdiff} cannot effectively display
changes to figures and tables, this document serves as a brief guide to our
major revisions.

Below, we highlight the most significant changes made in response to reviewer
input, including references to new visualizations that directly address the
points raised during review.

\subsection*{Better Justification around Finalizers}

\begin{blockquote} explicitly discuss and present the issues around
  finalizations semantics, as learned from other languages...\\

... This should take the form of a) explicit acknowledgement of the issue of
this as a solution (as per the JEP) b) explain why such an undesirable solution
is unavoidable (if it is) c) clearly explain how the problems with finalization
semantics relate to destructor semantics which Rust has already bought into
(i.e. exactly how much introducing this approach makes things worse or not).
\end{blockquote}

We have updated Section 4.1 to include a more detailed discussion of how
finalization semantics relate to destructor semantics in Rust in particular. We
offer a concrete example of an instance where finalizers can actually lead to
more drop methods being run than their equivalent destructor based counterparts
-- across panic unwind boundaries on different mutator threads.

We also update Section 4.2 to more explicitly address finalizer challenges
(such as latency) as they relate to JEP-421.

In Section 4 we mention that despite the issues surrounding finalization,
Rust's heavy reliance on destructor-based code, and Alloy's desire allow GC
interaction with these objects, finalization is inevitable.

\subsection*{Performance Evaluation}

\begin{blockquote}
include the non-BDW baseline in the evaluation and discuss the utility of this
approach w.r.t. that performance result. Ideally this would include a
discussion of how much of the performance deficit is likely redeemable via
"engineering", and how much is likely to be intrinsic (i.e. due to
unavoidable conservatism), so that we can get a better intuition of the
plausible "best case" for this approach.
\end{blockquote}

We have standardized all visualizations in Section 8 by normalizing against
appropriate baselines. Each experiment uses a specific baseline for comparison:
non-BDW allocator benchmarks for the \Egcrc experiment (RC comparison)
\jake{TODO link to figures}, naive Alloy without elision for the \Eelision
experiment \jake{TODO link to figures}, and barrier-free execution for the
\Epremopt experiment \jake{TODO link to figures}. All results are presented as
ratio relative to these baselines rather than raw values. For the interested
reader, we provide the raw values for each experiment in appendix tables.

We offer a discussion of the engineering cost in the prose in section \jake{TODO: section}

\subsection*{Conclusion}

\begin{blockquote}
add a discussion thats sums up the overall benefits/drawbacks of this
particular approach, possibly depending on specific scenarios/use cases, and
perhaps outlines directions for future work from a conceptional perspective,
i.e., possible design directions that could be investigated
\end{blockquote}


\end{document}
